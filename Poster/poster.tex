\NeedsTeXFormat{LaTeX2e}

\documentclass[portrait]{a0poster}

\usepackage{epsfig}
\usepackage{color}
\usepackage{transparent}
\usepackage{wallpaper}
\usepackage{amsmath}
\usepackage{amssymb}
\usepackage{calc}
\usepackage{bm}
\usepackage[utf8]{inputenc}
\usepackage{textcomp}
\usepackage{mathptmx}
\usepackage[scaled=.90]{helvet}
\usepackage{courier}
\usepackage{listings}

\hyphenation{re-so-na-tors na-no-tubes}

\pagestyle{empty}
\TileWallPaper{10.234cm}{11.594cm}{images/dokutitle}

\setlength{\parskip}{5pt plus 2pt minus 1pt}
\setlength{\parindent}{0pt}

\frenchspacing
\sloppy

\renewcommand{\familydefault}{\sfdefault}

% some color definitions
\definecolor{darkblue}{rgb}{0, 0, 0.5}
\definecolor{lightblue}{rgb}{0, 0, 0.8}
\definecolor{lightgray}{gray}{0.75}

\newcommand{\blue}{\color[rgb]{0.2,0.2,1}}
\newcommand{\lightblue}{\color{lightblue}}
\newcommand{\red}{\color{red}}
\newcommand{\darkgreen}{\color[rgb]{0,0.65,0}}

% Delft blauw 
\definecolor{background}{rgb}{1,1,1}
\definecolor{text}{rgb}{0, 0, 0.3}
\definecolor{heading}{rgb}{0, 0, 0.8}

\definecolor{invheading}{rgb}{0.8,0.8,1}

\pagecolor{black}

\lstset{
  basicstyle=\ttfamily,
  identifierstyle=\color[rgb]{0,0,0.5},
  stringstyle=\color[rgb]{0,0.5,0}\textit,
  commentstyle=\color{red}\textbf
}




% large heading in text
\newcommand{\heading}[1]{
  {\color{heading}\boldmath\textbf{\huge #1}}\\[\medskipamount]
}

% small heading in text
\newcommand{\smallheading}[1]{
  {\color{heading}\textbf{\Large #1}}\\
}

% very small heading in text
\newcommand{\verysmallheading}[1]{
  {\color{heading}\textbf{\large #1}}\\
}

% white bar across the column
\newcommand{\whitebar}{%
\vspace*{0.5cm}
\hspace*{-1.5cm}
\fcolorbox{white}{white}{\rule{30cm}{-0.7cm}}

\vspace*{1cm}
}

% white background for plots etc.
\newcommand{\whitebg}[1]{
 {
  \setlength{\fboxsep}{5mm}
  \setlength{\fboxrule}{1pt}
  \fcolorbox{black}{white}{\color{black}{#1}}
 }
}

% place a figure
\newcommand{\placefigure}[1]{ 
 {
  \vspace{-0.4cm}
  \begin{center}
  \hspace*{-1cm}\begin{minipage}{0.96\textwidth}
   \whitebg{
     \epsfig{#1,width=0.96\textwidth}
   }
   \vspace*{-1cm}
   {\\}
  \end{minipage}
  \end{center} 
  \vspace{-1cm}
 }
}

% place a listing
\newcommand{\placelisting}[1]{ 
 {
  \vspace{-0.2cm}
  \begin{center}
  \hspace*{-1cm}\begin{minipage}{0.96\textwidth}
   \footnotesize \renewcommand{\baselinestretch}{0.85}
   \whitebg{
   \parbox{0.97\textwidth}{
    \lstinputlisting[language=Perl]{#1}
   }
   }
   \renewcommand{\baselinestretch}{1}
   \vspace*{-1cm}
   {\\}
  \end{minipage}
  \end{center} 
  \vspace{-1cm}
 }
}

% place a listing
\newcommand{\placesmalllisting}[1]{ 
 {
  \vspace{-0.2cm}
  \begin{center}
  \hspace*{-1cm}\begin{minipage}{0.96\textwidth}
   \scriptsize \renewcommand{\baselinestretch}{0.8}
   \whitebg{
   \parbox{0.97\textwidth}{
    \lstinputlisting[language=Perl]{#1}
   }
   }
   \renewcommand{\baselinestretch}{1}
   \vspace*{-1cm}
   {\\}
  \end{minipage}
  \end{center} 
  \vspace{-1cm}
 }
}

% place a text
\newcommand{\placetext}[1]{ 
 {
  \vspace{-0.2cm}
  \begin{center}
  \hspace*{-1cm}\begin{minipage}{0.96\textwidth}
   \footnotesize \renewcommand{\baselinestretch}{0.85}
   \whitebg{
   \parbox{0.97\textwidth}{
    \lstinputlisting{#1}
   }
   }
   \renewcommand{\baselinestretch}{1}
   \vspace*{-1cm}
   {\\}
  \end{minipage}
  \end{center} 
 }
}




\begin{document}

\color{text}
\Large
%
\newlength{\seplength}
\newlength{\headerheight}
\newlength{\columnheight}
\newlength{\columnheighta}
\newlength{\columnheightb}
%
\setlength{\columnheight}{104cm}           % value for portrait 3-col mode
\setlength{\columnheighta}{\columnheight-6cm}
\setlength{\columnheightb}{\columnheight+6.5cm}
\setlength{\columnwidth}{0.31\textwidth}   % value for portrait 3-col mode
%
\setlength{\fboxsep}{10mm}
\setlength{\fboxrule}{0mm}
%
%
%
% the page header
%
\hspace*{-1cm}\begin{minipage}[t][][t]{\textwidth-2\fboxsep-8\fboxrule}
\color{invheading}
\begin{center}
{
\VERYHuge \vspace*{-0.6cm}
\textsf{\textbf{
Lab::Measurement -- measurement control with Perl
}}
}
\\[\baselineskip]
\begin{tabular}{ccc}
\parbox{5cm}{\vspace*{-1cm}
\begin{center}
\epsfig{file=images/logo-ur,height=4.5cm}
\end{center}
}
&
\parbox{70cm}{
\begin{center} 
{
\vspace*{-0.5cm}
\huge 
S. Reinhardt$^1$, C. E. Lane$^2$, C. Butschkow$^1$, A. Iankilevitch$^1$, A. 
Dirnaichner$^1$, 
and A. K. Hüttel$^1$
}
\\[\medskipamount]
\it 
\Large
$^1$Institute for Experimental and Applied
Physics, University of Regensburg, 93040 Regensburg, Germany\\
$^2$Department of Physics, Drexel University, 3141 Chestnut Street, 
Philadelphia, PA 19104, USA\\
\end{center}
\vspace*{0.5cm}
}
&
\parbox{5cm}{\vspace*{-1cm}
\begin{center}
\epsfig{file=images/logo-sfb689,height=4.5cm}
\end{center}
}
\end{tabular}
\end{center}
\end{minipage}
\vspace*{9mm}





%
% begin of first row first column
\fcolorbox{white}{background}{
\begin{minipage}[t][\columnheighta-2\fboxsep-2\fboxrule][t]
  {\columnwidth-2\fboxsep-2\fboxrule} \rule{0pt}{0pt}\\
  \begin{minipage}{\textwidth}
%
%

\vspace*{0.9cm}
\heading{Flexible measurement needed?!}

\vspace*{-1cm}
\begin{itemize}
\item
Tired of following your wires in square meters of LabVIEW diagrams?
\item
Tired of clumsy string handling and low-level driver functions
in your looong C program?
\item
Use a text processing language to manage your measurement! Use Perl!
\end{itemize}
\vspace*{0.5cm}

\placelisting{srs_read.pl}

\vspace*{2cm}
\vspace*{0.9cm}
\heading{Currently supported hardware}

\vspace*{-3mm}
\placefigure{file=images/hardware}
Hardware driver backends: 
\begin{itemize}
\item 
NI-VISA (both on MS Windows and on Linux) and all hardware supported by it
\item
LinuxGPIB and all hardware supported by it
\item
Linux USB-TMC kernel driver
 \item
Oxford Instruments IsoBus
 \item
TCP connection, generic network socket
 \item
Serial port, USB serial connection
\end{itemize}

\vspace*{1cm}
Growing number of high-level drivers (more are {\it very} easy to add):
{\large
\begin{itemize}
\item Multimeters: HP / Agilent / Keysight
\item DC sources: Yokogawa / Keithley
\item Lock-in amplifiers: Stanford Research / Signal Recovery
\item Temperature controllers: Lakeshore / Oxford Instruments
\item RF / microwave sources, spectrum analyzers, VNAs from Rohde \& Schwarz 
\item and many more...
\end{itemize}
}
\vspace*{1cm}
\heading{Key facts}
\vspace*{-2cm}
\begin{itemize}
 \item
 Open source / free software\\[-5cm]
 \item 
 http://www.labmeasurement.de/
 \hfill\epsfig{file=images/qrcode-labmeasurement,width=5cm}
 \item
 License: same as Perl (GPL-1+ or Artistic)
 \item
 Releases on CPAN, development on Github
 \item
 Contributors and cooperations welcome!
\end{itemize}






%
%
%
\end{minipage}
%
\end{minipage}}
% end of first row first column
%
%
%
%
% begin of first row second column
\fcolorbox{white}{background}{
\begin{minipage}[t][\columnheighta-2\fboxsep-2\fboxrule][t]
  {\columnwidth-2\fboxsep-2\fboxrule} \rule{0pt}{0pt}\\
  \begin{minipage}{\textwidth}
%
%
%
%

\vspace*{0.9cm}
\heading{Real world measurement}

\vspace*{-1cm}
\begin{itemize}
 \item 
 Nested sweeps: gate voltage $V_\text{g}$, bias voltage $V_\text{sd}$
 \item
 For each point, read out multimeter, write values to data file
 \item 
 Regularly updated ``live'' color plot
\end{itemize}


\vspace*{0.5cm}
\placesmalllisting{xpress_2d_example.pl}

\vspace*{3cm}

\heading{Output files}
\vspace*{-1cm}
\placesmalllisting{ls.txt}
\vspace*{2cm}
\begin{itemize}
\item Archival copy of the measurement script
\item Measured data, tab-separated gnuplot format
\item Live plot at end of measurement as png image
\end{itemize}




%
%
%
%
\end{minipage}
\end{minipage}}
% end of second column
%
%
%
%
%
%begin third column
\fcolorbox{white}{background}{
\begin{minipage}[t][\columnheighta-2\fboxsep-2\fboxrule][t]
  {\columnwidth-2\fboxsep-2\fboxrule} \rule{0pt}{0pt}\\
  \begin{minipage}{\textwidth}
%
%
%
%
\vspace*{0.9cm}

\heading{Internal architecture}

\vspace*{-3mm}
\placefigure{file=images/structure}

\begin{itemize}
\item
Modular structure. Easy to extend with new instrument drivers and connection
types.
\item
Abstract IO layer, makes instrument drivers independent of hardware backends.
\end{itemize}
\vspace*{0.9cm}
\heading{Recent improvements}
\vspace*{-2cm}
\begin{itemize}
\item {\lightblue New drivers}: R\&S FSV spectrum analyzers, R\&S ZVA and ZVM 
vector
  network analyzers, HP33120 function generator, Tektronix TDS2024
  oscilloscope
\item {\lightblue Automated testing} of instrument drivers and high-level 
functionality via
  mock connection objects; continuous integration testing on both Linux (Travis 
  CI) and Windows (Appveyor)
\item 
  {\lightblue New framework for drivers and connections} based on 
\textbf{Moose} --- de-facto
  standard for Perl5, enabling modern object oriented programming
  \begin{itemize}
  \item Less boilerplate code, create classes in a descriptive way
  \item Use \textbf{roles} for safe and efficient sharing of functionality
    between drivers
  \item Organize SCPI subsystems into roles, which can be used by multiple
    drivers --- see e.g. the new R\&S FSV spectrum analyzer driver:
\vspace*{4mm}
\placesmalllisting{fsv-roles.pl}
\vspace*{1.5cm}
  \item Extensive type system; easy validation of user provided 
 function parameters.
  \end{itemize}
  

\item 
Access to all of gnuplot's {\lightblue plot and curve options} via 
PDL::Graphics::Gnuplot --- both for live plots (qt, x11, wxt) and hardcopies 
(png, jpeg, svg, tikz, pdfcairo, ...)
\end{itemize}

\vspace*{2cm}
\heading{Outlook}
\vspace*{-2cm}

\begin{itemize}
\item Implement more drivers, connections, and XPRESS functionality with
  Moose
\item Build portable, lightweight USB-TMC connection based on libusb
\end{itemize}


\end{minipage}

\end{minipage}}  % end of first row third column
%
%
%

\vspace*{0.5cm}
\begin{center}\color{invheading}\large
\hspace*{-4cm}We gratefully acknowlegde funding by the DFG via the Emmy Noether grant
Hu1808/1-1 ``Carbon nanotubes as electronical and nano-electromechanical hybrid systems in
the quantum limit'',
\\ \hspace*{-4cm}
the collaborative research centre SFB 689 ``Spin phenomena in reduced dimensions'', and
the graduate research school GRK 1570  ``Electronic properties of
carbon based nanostructures''.
\end{center}

\end{document}
